\documentclass[11pt,a4paper]{article}

\usepackage{amsmath,amssymb}
\usepackage{graphicx}
\usepackage{epsfig}
\usepackage{url}
\usepackage{blindtext}
\usepackage{scrextend}
\usepackage{tasks}
\usepackage[english]{babel}
\usepackage[utf8]{inputenc}
\addtokomafont{labelinglabel}{\sffamily}



\begin{document}

\title{2 Rewirewing Networks}
\author{Evan Huygen, Marcel Vo{\ss}hans}
\maketitle

\begin{abstract}
This report is about networks and more precise about the graph theory. A graph has edges and nodes, the edges connect the nodes with each other, the whole model is called a graph. The average path length (APL) describes the average distance from every single node to every other. In a symmetric graph, all nodes have an equal path length. In this report, we found out that the more rewiring we implement, the more decreases the APL.
\end{abstract}

\tableofcontents

\section{Introduction}
INTRODUCTION GOES HERE...
 \subsection{Setup}
The program Mathematica from Wolfram is a very powerful tool to create Models like the regarded graphs. So, for this model is used the latest (2017.10.21) version of Mathematica (v11.1.1.0). The kernel runs on a Windows 10- 64bit OS.
 \subsection{Realization in Mathematica}
The entire project is divided into two basic programs while one simulates a specific/ random case of rewiring and the other analyses an entire set of cases. So, the analysing program extends the specific case program. 
 \subsubsection{The Specific Case}
\begin{labeling}{step}
\item [a)] \textbf{Basic Setup} - The  basic idea of a graph is to use a incidence matrix. The columns describe the edges and the rows display the nodes. For a better handling, will be the following defined: every edge belongs to one node (because a connected graph is necessary) and every edge goes to the belonging node e.g.
Edge \(1\) belongs to node \(1\) (it comes from elsewhere and goes to \(node 1\))
Referring to this rule the incidence matrix become to:
\begin{figure}[h]
\centering
\includegraphics[width=0.15\textwidth]{basicIncidenceMatrix.png}
\caption{An adjacent Matrix (without rewiring) and \(n=5\)}
\label{fig:basicMatrix}
\end{figure}
It follows the graph. Counter clockwise labelled:
\begin{figure}[h]
\centering
\includegraphics[width=0.25\textwidth]{basicGraph.png}
\caption{A connected, simple Graph with \(n=5\). Belongs to Figure \ref{fig:basicMatrix}}
\end{figure}
\item [b)] \textbf{Matrix Manipulation} - In the second step follows the randomization. An edge can (physically) be connected to just two nodes so every column can hold only two \(1\). In a row can be \(1\) as much as possible but at least one (to be at least connected to the rest of the network with its belonging edge). At first, we pick randomly an edge then is it needed to find the changeable node (because every edge belongs to a node). Now can we randomly determine a new node to connect. But because of the randomization are there some checks for the new node needed:
\begin{itemize}  
\item Is the new node, the old (already connected) node? 
\item Is the new connection still existing? (simple graph) 
\item Is it still a connected graph? (otherwise: rollback)
\end{itemize}
If one of these checks fail, must the new node (randomly) be picked again. The number of rewiring depends on the given percentage to be rewired for e.g.: (based on the example from [1])
\begin{figure}[h]
\centering
\includegraphics[width=0.3\textwidth]{rewiredMelanieMitchell.png}
\caption{A network with \(n=60\) and a rewiring of \(5\%(3)\) refered to \cite{melanie}}
\end{figure}
In this program is an auto-random mode implemented, if it is deactivated the rewiring can be entered manually.
\item [c)] \textbf{Output} - The Output handles the conversion of the incidence matrices and organises the display for example the APL which is realized with the built-in \(MeanGraphDistance[]\) function.
\end{labeling}
\subsubsection{The Analysing Case}
This program extends the first program (the specific case). The entire randomization will be repeated \(p\) times (precision) and every APL will be stored in a list. After p times it stores the average of all APL with this given value of rewiring (in percentage). This process will be repeated twenty times with a respectively increment of \(1\%\) rewiring per repetition. So, a whole analysing run shows the average of the APL over an area of \(20\%\) starting at \(5\%\) e.g.:
\begin{figure}[h]
\centering
\includegraphics[width=0.6\textwidth]{analysingAPL.png}
\caption{A network with \(n=60\) a rewiring from \(5-25\%\) and a repetition of \(p=100\)}
\end{figure}
\subsection{Average Path Length}
In the literature is in general the APL described by the following formula\cite{apl}:
\begin{align}
APL=\frac{1}{n\cdot(n-1)} \cdot \sum \limits_{i\neq j} d(v_i,v_j)
\end{align}
While \(n\) describes the number of nodes and \(d(v_i,v_j)\) the shortest distance between two nodes.

\section{Results}
RESULTS GOES HERE...
\section{Discussion}
Based on the paper of shows this model the basic dynamic of a rotating system like our solar system or even a billiard table, where the billiard ball moves around a triangle. Basically, is it claimable that the ball will stuck in a loop, what is not obvious to see. In case of our solar system is it also claimable that the planets can’t escape from the central point. But there are some cases when the loop-length raises up to infinity so it is hard to make a statement with this model because we don’t know the start of the planet or the opportunity that it falls into an in-line position to the center (if we assume that the sun is a triangle). Thinking forward and to raise the number of corners to infinity to model a circle (like the sun in a 2D model), remains the question of an in-line position of one of our solar-system-planets. But this model can show just the fact of a loop building trajectory and the exception of a resonating opportunity. It depend on the position to the triangle and the distance to the next corner (loop-length). Conclusive can we say, that the ball won't escape from the triangle (by assuming constant speed and no outside influences). In the solar system, is the weightlessness for constant speed and the gravity as pulling force, so it is also assumable that no planet ever will leave our solar system. When the planet is in a loop around the Sun (and would have a linear movement).

\listoffigures

\begin{thebibliography}{3}
 \bibitem{doc}Wolfram Language \& System, \url{http://reference.wolfram.com/language/}, 2017.10.20.
 \bibitem{melanie} Melanie Mitchel, "Complexity - A Guided Tour", (2009), Chapter 15.
\bibitem{apl}Albert, R\'eka and Barab\'asi, Albert-L\'aszl\'o, "Statistical mechanics of complex networks", p.65, \url{https://doi.org/10.1103/RevModPhys.74.47} 
\end{thebibliography}

\end{document}
